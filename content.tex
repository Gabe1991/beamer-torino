% DO NOT COMPILE THIS FILE DIRECTLY!
% This file is included by the other .tex files.

\begin{frame}[t,plain]
\titlepage
\end{frame}

\begin{frame}[t]
\frametitle{Lorem ipsum dolor sit amet}

\begin{itemize}
\item For documentation see the following pages
\item Consectetur adipisici elit
\item Sed eiusmod tempor incidunt ut labore et dolore magna aliqua
  \begin{itemize}
  \item Ut enim ad minim veniam
    \begin{itemize}
    \item Quis nostrud exercitation
    \item Ullamco laboris nisi
    \end{itemize}
  \item Ut aliquid ex ea commodi consequat
  \end{itemize}
\item Quis aute iure reprehenderit in voluptate velit esse cillum dolore eu
      fugiat nulla pariatur
\item Excepteur sint obcaecat cupiditat non proident, sunt in culpa qui
      officia deserunt mollit anim id est laborum
\end{itemize}

\end{frame}

\begin{frame}[t]{De Finibus Bonorum et Malorum}
\framesubtitle{About The Purposes of Good and Evil}

Sed ut perspiciatis, unde omnis iste natus error sit voluptatem accusantium
doloremque laudantium, totam rem aperiam eaque ipsa.

\begin{itemize}
\item Ut aliquid ex ea commodi consequat
\item Sed eiusmod tempor incidunt ut labore et dolore magna aliqua
  \begin{itemize}
  \item Ut enim ad minim veniam
  \item Quis nostrud exercitation ullamco laboris nisi
  \end{itemize}
\item Consectetur adipisici elit
\item Quis aute iure reprehenderit in voluptate velit esse cillum dolore eu
      fugiat nulla pariatur
\item Excepteur sint obcaecat cupiditat non proident, sunt in culpa qui
      officia deserunt mollit anim id est laborum
\end{itemize}

\end{frame}

\begin{frame}[t,fragile]
\frametitle{Alternative title page}
\begin{itemize}
\item A fancy title page can be enabled with the \verb!alternativetitlepage! option
\item You can put a logo in the title page, just pass the file name with the
      \verb!titlepagelogo! option
\item Remember to use a plain and top-aligned frame when using alternative title
      pages:\\
      \vskip1ex
      \verb!\begin{frame}[t,plain]!\\
      \verb!\titlepage!\\
      \verb!\end{frame}!
\end{itemize}
\end{frame}

\begin{frame}[t,fragile]
\frametitle{Watermarks}
\begin{itemize}
\item A watermark can be shown in the bottom right corner of frames
\item The \verb!watermark! option is the name of the image file
\item The \verb!watermarkheight! option is the height of the watermark in the frame
  \begin{itemize}
  \item Not the original image height
  \item If you are displaying the image at a different size from the original you
        have to use the \verb!watermarkheightmult! option
  \end{itemize}
\end{itemize}

\end{frame}

\watermarkoff
\begin{frame}[t,fragile]
\frametitle{Frame without watermark}

\begin{itemize}
\item You may want to disable the watermark on some frames
  \begin{itemize}
  \item An image could cover the watermark causing an ugly effect
  \end{itemize}
\item The \verb!\watermarkoff! command can be used to disable the watermark for the following frames
\item The \verb!\watermarkon! command restore the normal watermark for the following frames
\item If you did not specify a watermark, nothing happens
\end{itemize}

\end{frame}
\watermarkon
